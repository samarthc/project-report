\documentclass{article}
\usepackage[utf8]{inputenc}
\usepackage{hyperref}

\def\code#1{\texttt{#1}}

\title{
PClub Project Report
\\
Ethical Hacking
}
\author{team}
\date{May 2016}

\usepackage{natbib}
\usepackage{graphicx}

\begin{document}

\maketitle

\newpage 

\tableofcontents

\newpage
\section*{Member 1: Tanmay Seth}
\subsection*{Basics Completed}
\begin{itemize}
    \item \textbf{ basic html, css, js, sql}
    \item \textbf{basics of vim text editor}
    \item \textbf{cyber security course on coursera first week}
\end{itemize}
\subsection*{Challenges completed}
\begin{itemize}
    \item \textbf{hts basic level 10}
    \item \textbf{bandit level 1 (recently started and currently working on)}
    \item \textbf{natas level 3}
\end{itemize}
\subsection*{Others}
\begin{itemize}
    \item \textbf{made my website (not uploaded it yet)}
\end{itemize}
\section*{\textit{Timeline of work done}}
\subsection*{\textit{till 22nd May}}
\begin{itemize}
    \item \textbf{covered basics of html, css, js, sql}
\end{itemize}
\subsection*{\textit{till 26th May}}
\begin{itemize}
    \item \textbf{challenges of hack this site till basics level 10}
    \item 
\textbf{started creation of my homepage}
    \item \textbf{basics of vim editor}
\end{itemize}
\subsection*{till 29th May}
\begin{itemize}
    \item \textbf{created my homepage}
    \item \textbf{first week of cyber security course on coursera}
\subsection*{in future}
\begin{itemize}
    \item \textbf{attemp more challenges and and levels.}
\end{itemize}

\newpage
\section*{Member 2: Samarth Chawla}
\subsection*{Summary of challenges completed}

\paragraph{\href{http://overthewire.org/wargames/bandit/}{Bandit}}
The bandit wargame only focuses on \textit{teaching} command line utilities that might be useful in exploits, not actual exploits
\begin{itemize}
    
    \item \textbf{Levels 0 - 6: } ssh and basics of command line; \code{ls, cat, file, find} \footnote{Reading up on \code{find} was useful for me, as initially I was using \code{ls} to \code{grep} for non-executables, then stripping out filenames and piping to \code{du} and \code{file}}
    
    \item \textbf{Levels 7 - 12: } matching text with \code{grep}, finding a uniquely string with \code{sort} and \code{uniq}, decoding base64 data with \code{base64}, deciphering a ROT13 cipher with \code{tr}, uncompressing a hexdump with \code{xxd} and \code{tar}
    
    \item \textbf{Levels 13 - 16: } working with network protocols; ssh using a private key, using \code{telnet} and \code{openssl} to exhange data, using \code{nmap} to scan for open ports
    
    \item \textbf{Level 17: } used \code{diff} to compare two files and extract the password
    
    \item \textbf{Level 18: } used \code{scp} to transfer files from an ssh server that logs you out immediately on logging in due to a modified .bashrc, thus bypassing the shell on the remote PC
    
    \item \textbf{Levels 19 - 20: } learned about setuid, learned to start network daemons from the command line using \code{nc}
    
    \item \textbf{Levels 21-23: } learned about \code{cron}; in level 23, abused a cronjob running all scripts from a particular directory to copy the password to a particular directory

\end{itemize}

\paragraph{\href{http://overthewire.org/wargames/bandit}{Natas}}

\begin{itemize}
    
    \item \textbf{Levels 0 - 2: } Password commented in source code of levels 0 and 1, simply opened the developer panel to access the password; level 2 had an image from subdirectory \code{/files}, and the index of \code{files} contained an unprotected file with the password
    
    \item \textbf{Level 3: } Comment in source code hinting at Google not being able to find the password. Had to look up the solution after a lot of trying, a directory listed in the site's \code{robots.txt} had the password
    
    \item \textbf{Level 4: } Had to again seek help (from Prannay this time) as I could not figure out what to do after a lot of trying. Tried \code{window.history.pushState("http://natas5.natas.labs.overthewire.org")} first, but found I had to edit the HTTP request header before it got sent, and did so using the Tamper extension on Chromium.
    
    \item \textbf{Level 5: } Edited the loggedin cookie using \code{document.cookie="loggedin=1"}
    
    \item \textbf{Level 6: } Found the secret from \code{/includes/secret.inc}
    
    \item \textbf{Level 7: } Modified \code{href} of link to \code{/etc/natas_webpass/natas8}
    
    \item \textbf{Level 8: } The secret is now encoded into base64, then reversed and converted to hex. Ideally, I would have written either a PHP script or something like a bash or python script, but since I wasn't familiar enough with either, I wrote a c program to convert the hex into ASCII, then ran the output through the unix programs \code{rev} and \code{base64_decode}.
    
    \item \textbf{Level 9: } Injected \code{; cat /etc/natas_webpass/natas10} into the PHP script to access the password.
    
    \item \textbf{Level 10: } Since ";|&" are now filtered, (after floundering about for a long while) I input \code{-e '.' /etc/natas_webpass/natas11 #} to output every line from \code{/etc/natas_webpass/natas11}
    
\end{itemize}

\paragraph{\href{http://hackthissite.org/missions/basic}{Hackthissite basic missions}}, except the last basic mission

\paragraph{\href{http://hackthissite.org/missions/javascript}{Hackthissite javascript missions}}

\subsection*{Future Tasks: Upcoming week}
\begin{itemize}
    \item Begin with Week 1 of the Coursera course on Software Security
    \item Move ahead with Natas
    \item Learn PHP
\end{itemize}

\newpage
\section*{Member 3: Prannay Khosla}
\subsection*{Work done till now}
\begin{itemize}
	\item Completed till Level 17 NATAS, Level 10 of HTS Basics and all Javsascript challenges on HTS.
	\item Completed till week 2 of the coursera course of Software Security by the  University of Maryland. 
	\item Studied basic of Architecture and 3 address code and a little bit about processors.
\end{itemize}
\paragraph{Work was done in a discontinous manner without a proper timeline, since more time was spent on learning and developing rather than actual hacking.}
\begin{itemize}
	\item \textit{NATAS Level 0 to Level 2} : Basic HTML source code checking, browser tools etc
	\item \textit{NATAS Level 3 to Level 5} : Basic scripting and request tampering
	\item \textit{NATAS Level 6 to Level 8} : Basic PHP injection and reverse engineering from source code. 
	\item \textit{NATAS Level 9 to 11} : Injecting bash commands from PHP and reverse engineering from known encryptions.
	\item \textit{NATAS Level 12 to 13} :  Injecting files that run system commands into the website to hack data. 
	\item \textit{NATAS Level 14 to 17} : SQL injection using arguments, sometimes requiring python based brute forcing. 
	\item \textit{HTS Basic Level 1 to 10} : Basic PHP, SQL injection and command line injection
	\item \textit{HTS Javascript Level 1 to 7} : All challenges were trivial except the last which required brute forcing from the console. 
\end{itemize}

\paragraph{Future work:}
\begin{itemize}
	\item Work forward with system architecture and reverse engineering binaries.
	\item Move on to Narnia wargames for hacking
	\item Start blogging about good challenges.
\end{itemize}


\newpage
\section*{Member 4: Jaismin Kaur}

\newpage
\section*{Member 5: Bhuvi Gupta}
\subsection*{Work done till now}
\paragraph{}
As a part of the ethical hacking project, I have completed the following:
\begin{itemize}
    \item Basics of HTML, CSS(Moderate), JS, SQL, PHP(beginner)
    \item Completed Week 2 of the coursera course on Software Security, Project 1 is still pending
    \item Level 18 of bandits, level 10 of natas, basic challeges in HTS level 10
\end{itemize}

\paragraph{Time line of work done}
\subparagraph{\textit{Till 20th May}}
    \begin{itemize}
    \item Basics of HTMl, CSS, JS, SQL 
    \item Udacity's course on git and git hub 
    \item Basics of vim 
    \item Week 1 of Software Security(Coursera), excluding project 
    \item Creation of home page directory
    \end{itemize}
\subparagraph{\textit{TILL 25TH MAY}}
\subparagraph{Bandits level 15}
    \textit{First Five Levels}
    \begin{itemize}
    \item Basic terminal commands such as:ls, la, escaping special characters, du, file, cat, find
    \item Connect to a server using ssh
    \end{itemize}
    \textit{Next Five Levels (6-10)}
    \begin{itemize}
    \item Terminal commands like grep, tar, bzip2, gzip, xxd, base64, diff, sort, uniq, tr
    \end{itemize}
    \textit{Next Five Levels(10-15)}
    \begin{itemize}
    \item Terminal commands such as:tr, telnet, nmap, openssl
    \item Connect to a server using ssh key
    \end{itemize}
\subparagraph{Natas level 5}
    \begin{itemize}
    \item Basic intro to using the developer panel
    \item robots.txt
    \item Exploiting vulnerabilities in functions which mail the flag on a specified mail id
    \end{itemize}    
\subparagraph{Hack this Site Java script challenges}
    \begin{itemize}
    \item Completed all levels except for last
    \end{itemize}    
    
\subparagraph{\textit{Till 28th May}}
\subparagraph{Bandits level 19}
    \textit{Level 16-19}
    \begin{itemize}
    \item Terminal commands such as:nmap, diff
    \end{itemize}
    
\subparagraph{Natas level 6-10}
    \begin{itemize}
    \item Accessing password by exploiting general terminal commands such as grep and cat.
    \item Using tamper to change referer 
    \item Basic intro to cookies
    \item Basic intro to php
    \end{itemize}    
\subparagraph{Hack this Site Basic Challenges}
    \begin{itemize}
    \item Increased familiarity with web hacking
    \end{itemize} 
\subparagraph{Week 2 of coursera course}
    Basic knowledge about the following:
    \begin{itemize}
    \item Stack canaries
    \item ROP
    \item Return to libC
    \end{itemize} 
    
\subsection*{Future Tasks:Upcoming week}
\begin{itemize}
    \item Move ahead with the coursera course
    \item Read more about assembly language
    \item install ida/hopper
    \item Attempt Narnia challenges
\end{itemize}
\end{document}
